\documentclass{article}
\usepackage{tikz}
\newcommand*\circled[1]{\tikz[baseline=(char.base)]{\node[shape=circle,draw,inner sep=1pt] (char) {#1};}}
\usepackage{fancyhdr}
\usepackage{newtxmath}
\usepackage{enumitem}
\usepackage{stix}
\pagestyle{fancy}
\usepackage{listings}


\title{Assignment 11 - Question 6 - Complexity Analysis} 
\author{Omar Abdelhady \\ \\900222873}
\date{\today}

\fancyhead{}
\fancyhead[L]{Assignment 11 - Question 6 - Complexity Analysis}

\begin{document}
\maketitle

\newpage

\subsection*{Complexity Analysis for the Backtracking Algorithm}
\subsubsection*{Time Complexity}

The time complexity of the provided backtracking algorithm can be analyzed as follows:

\begin{itemize}
    \item The algorithm explores all possible ways to divide the input set into two subsets, considering each element for either subset.
    \item For a set of size $n$, there are $2^n$ possible ways to assign each element to a subset.
\end{itemize}

The recurrence relation for the algorithm is:
\[
T(n) = 2T(n-1) + 1
\]
Expanding this recurrence:
\begin{align*}
T(n) &= 2T(n-1) + 1 \\
T(n-1) &= 2T(n-2) + 1 \\
T(n-2) &= 2T(n-3) + 1 \\
&\vdots \\
\end{align*}

Continuing this pattern, we get:
\[
T(n) = 2^n T(0) + 2^{n-1} + 2^{n-2} + \ldots + 2^0
\]

The sum of the geometric series simplifies to:
\[
T(n) = 2^n T(0) + (2^n - 1)
\]

Since $T(0)$ is a constant, we can denote it as $c$. Thus, the time complexity is:
\[
T(n) = O(2^n)
\]

\subsubsection*{Space Complexity}
The space complexity of the algorithm can be analyzed as follows:
\begin{itemize}
    \item The algorithm uses a recursive approach, which requires space for the call stack.
    \item The maximum depth of the recursion is $n$, leading to a space complexity of $O(n)$ for the call stack.
    \item Additionally, the algorithm uses space for the subsets and other variables, but these are bounded by $O(n)$ as well.
    \item Therefore, the overall space complexity is $O(n)$.
\end{itemize}

\textbf{Conclusion:} The time complexity of the backtracking algorithm is $O(2^n)$, and the space complexity is $O(n)$.

\newpage

\section*{Complexity Analysis for the DP Algorithm}

\subsubsection*{Time Complexity}
The time complexity of the dynamic programming function can be analyzed as follows:
\begin{itemize}
    \item The DP state is defined by three parameters: $n$ (number of elements left), $sumCalculated$ (current sum of one subset), and $subsetSize$ (number of elements in the subset).
    \item $n$ can go from $0$ to $N$ (where $N$ is the size of the input array).
    \item $sumCalculated$ can range from $0$ to $S$, where $S$ is the total sum of all elements in the array.
    \item $subsetSize$ can go from $0$ to $N$.
    \item The memoization table has size $O(N \cdot S \cdot N)$.
    \item For each state, the function does $O(1)$ work (two recursive calls and a min).
    \item Therefore, the overall time complexity is $O(N^2 \cdot S)$, where $S$ is the total sum of the array.
\end{itemize}

\subsubsection*{Space Complexity}
The space complexity of the dynamic programming function is:
\begin{itemize}
    \item The memoization table uses $O(N^2 \cdot S)$ space.
    \item The recursion stack uses $O(N)$ space (maximum depth).
    \item Therefore, the total space complexity is $O(N^2 \cdot S)$.
\end{itemize}

\textbf{Conclusion:} The time and space complexity of the dynamic programming function are both $O(N^2 \cdot S)$, where $N$ is the number of elements and $S$ is the total sum of the array.

\end{document}
